\chapter{Cloud computing}


\section{What is cloud computing?}



\subsection{Characteristics}

Characteristics [2]:

\begin{itemize}
	\item \textbf{On-demand self-service} : A consumer can unilaterally provision computing capabilities, such as server time and network storage, as needed automatically without requiring human interaction with each service provider;
	\item \textbf{Broad network access} : Capabilities are available over the network and accessed through standard mechanisms that promote use by heterogeneous thin or thick client platforms, e.g., mobile phones, tablets, laptops, and workstations;
	\item \textbf{Resource pooling} : The provider's computing resources are pooled to serve multiple consumers using a multi-tenant model, with different physical and virtual resources dynamically assigned and reassigned according to consumer demand. There is a sense of location independence in that the customer generally has no control or knowledge over the exact location of the provided resources but may be able to specify location at a higher level of abstraction, e.g., country, state, or datacenter. Examples of resources include storage, processing, memory, and network bandwidth;
	\item \textbf{Rapid elasticity} : Capabilities can be elastically provisioned and released, in some cases automatically, to scale rapidly outward and inward commensurate with demand. To the consumer, the capabilities available for provisioning often appear to be unlimited and can be appropriated in any quantity at any time;
	\item \textbf{Measured service} : Cloud systems automatically control and optimize resource use by leveraging a metering capability at some level of abstraction appropriate to the type of service (e.g., storage, processing, bandwidth, and active user accounts). Resource usage can be monitored, controlled, and reported, providing transparency for both the provider and consumer of the utilized service;
\end{itemize}



Outsourcing requires trust from clients.


A number of advantages are associated with cloud computing [2]:
\begin{itemize}
	\item Accelerated deployment of new applications without consuming enterprise's existing IT resources;
	\item Reduced capital requirements for up-front IT investments;
	\item Flexibility to meet sudden changes in demand peaks and troughs;
	\item Capability to match current and future demand;
	\item Significant cost savings through centralization when scale of enterprise IT resources << cloud provider;
	\item Data sharing and collaboration for multi-party processes. More economical and faster to deploy centrally;
\end{itemize}






\subsection{Business models}

Cloudcomputing is typically a pay per use model for on-demand, convenient access to shared pool of computing resources, e.g., storage, CPU, network, and applications. There are three basic types of services in the cloud computing model [2]:
\begin{enumerate}
	\item \textbf{Infrastructure as a service (IaaS)} : virtual machine with processing, storage and networking;
	\item \textbf{Platform as a service (PaaS)} : development platform and associated tools, e.g., PHP, .NET, Java;
	\item \textbf{Software as a service (SaaS)} : Zero-install, online applications, e.g., CRM, document processing platforms, application specific record management etc.;
\end{enumerate}


\subsubsection{Infrastructure as a Service (IaaS)}

"The capability provided to the consumer is to provision processing, storage, networks, and other fundamental computing resources where the consumer is able to deploy and run arbitrary software, which can include operating systems and applications. The consumer does not manage or control the underlying cloud infrastructure but has control over operating systems, storage, and deployed applications; and possibly limited control of select networking components (e.g., host firewalls)." [3]


\subsubsection{Platform as a Service PaaS}

"A cloud service model that provides the consumer the capability to deployonto the cloud infrastructure consumer-created or acquired applicationscreated using programming languages, libraries, services, and tools supported by the provider.1The consumer does not manage or control the underlying cloud infrastructureincluding network, servers, operating systems, or storage, but has control over the deployed applications and possibly configuration settings for the application-hosting environment." [3]



\subsubsection{Software as a Service SaaS}

"The capability provided to the consumer is to use the provider's applications running on a cloud infrastructure(*). The applications are accessible from various client devices through either a thin client interface, such as a web browser (e.g., web-based email), or a program interface. The consumer does not manage or control the underlying cloud infrastructure including network, servers, operating systems, storage, or even individual application capabilities, with the possible exception of limited user-specific application configuration settings." [3]

\subsection{Value levels of cloud computing}

In [4] three value levels of cloud computing are discussed:
\begin{itemize}
	\item \textbf{Utility level} : Enterprises can benefit from lower costs and higher service levels through the availability of elastic computing resources and pay-per-use models;
	\item \textbf{Process transformation level} : Enterprises can introduce new and improved business processes by leveraging the common and scalable assets and collaborative potential of cloud computing;
	\item \textbf{Business model innovation level} : New business models can be created by linking, sharing and combining resources using cloud computing in an entire business ecosystem;
\end{itemize}



%\section{Technologies}
%
%Cloud computing involves a wide range of possible implementations and techniques; some of these are discussed in the following paragraphs.
%
%Centralized Approaches to online social networking
%
%The Need for Scalability
%Ultra large scalability of systems.
%Highly distributed content production and querying.
%Many users:
%- Business queries.
%- End-user actions.
%
%as system complexity increases, our ability to reason about that system decreases => increase the granularity at which we reason, to minimize complexity.
%
%
%New Communication Paradigms
	%- Protocol Buffers (PB)
	%- Publish-Subscribe dissemination of distributed events in real time
%
%New Models of Computation
	%- MapReduce
%
%Increasing scale further
	%- Hadoop
		%* NameNode
		%* DataNode
	%- Hive






\section*{References}

\begin{enumerate}[1]
	\item G. Coulouris, J. Dollimore, T. Kindberg and G. Blair, "Distributed Systems: Concepts and Design (5th Edition)", M. Horton, Red., Addison-Wesley, 2011, p. 1063.
	\item Wouter Joosen, 2013, "Perspectives on Cloud Computing", iMinds-DistriNet, KU Leuven
	\item P. Mell and T. Grance, 2011, "The NIST Definition of Cloud Computing", online, available at: http://csrc.nist.gov/publications/nistpubs/800-145/SP800-145.pdf
	\item D. Dean and T. Saleh, 2009: "Capturing the Value of Cloud Computing: How Enterprises Can Chart Their Course to the Next Level", BCG -http://www.bcg.be/documents/file34246.pdf
\end{enumerate}